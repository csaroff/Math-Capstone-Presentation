% An Overview of Lie Algebra's
% Adapted from Justin Ryan's "Examples of Lie Algebras"
% Chaskin Saroff and Alexander Jansing	
% last changed: 29 April 2015
% feel free to make any improvements/changes you wish

\documentclass[9 pt]{beamer}
% \usepackage{default}
\usepackage{lmodern}
\usepackage{amsmath,amsfonts,epsfig,pgf} % ,graphicx

% choose your theme
 \usetheme{Warsaw} % Warsaw, Copenhagen, Darmstadt, Madrid, Singapore, etc...

% custom SUNY Oswego color scheme
\definecolor{oswego}{rgb}{0.15,0.4,0.15}
\setbeamercolor*{palette primary}{fg=white, bg=oswego}
\setbeamercolor*{palette sidebar primary}{fg=black, bg=oswego}
\setbeamercolor{block title}{bg=black,fg=white} % bg=background, fg= foreground
\setbeamercolor{block body}{bg=oswego,fg=black} % bg=background, fg= foreground
\setbeamercolor{alerted text}{fg=oswego}
\usecolortheme[named={oswego}]{structure}
\def\today{\number\day\space\ifcase\month\or
   January\or February\or March\or April\or May\or June\or
   July\or August\or September\or October\or November\or December\fi
   \space\number\year}

% something I found to get alert blocks in the Oswego color scheme
\newenvironment<>{lakeblock}[1]{%
  \begin{actionenv}#2%
      \def\insertblocktitle{#1}%
      \par%
      \mode<presentation>{%
\setbeamercolor{block title}{fg=white,bg=black}
       \setbeamercolor{block body}{fg=white,bg=oswego}
            }%
      \usebeamertemplate{block begin}}
    {\par\usebeamertemplate{block end}\end{actionenv}}

% Oswego State logo on every page
%\logo{\pgfputat{\pgfxy(-10.5,0.35)}{\pgfbox[center,center]
%{\includegraphics[height=1.35cm]{oswego_logo_horiz_grn}}}}

% commutative diagrams with XY-pic
\usepackage[all]{xy}
\SelectTips{cm}{}
% make \mathscr, TeX \cal, and Euler script *all* available
% (notice the new command names to avoid overlap and/or confusion)
\usepackage{mathrsfs}
\let\rscr=\mathscr % use \rscr{} for Ralph Smith fancy script
\let\mathscr=\relax
\let\mcal=\mathcal % use \mcal{} for TeX \cal script
\usepackage{eucal}
\let\escr=\mathcal % use \escr{} for Euler script
\let\mathcal=\relax
% a better "bar" thanks to Donald Arsenau -- see \pbar infra
\usepackage{accents}

% title page information
\title[Recovering the Metric - An Overview of Lie Agebra's]{Recovering the Metric - An Overview of Lie Algebras}
\author[A.\, Jansing \,C.\, Saroff]{Alexander Jansing,  Chaskin Saroff}
\institute[SUNY Oswego]{Oswego State University\\ Department of Mathematics}
\date{\today}

% new math commands
\newcommand{\at}[1]{\emph{\alert{#1}}}
\newcommand{\ad}[1]{\text{ad}_{#1}}
\newcommand{\add}[1]{\ad{#1}^\dagger}
\newcommand{\br}[2]{\left[ #1, #2 \right]}
\newcommand{\bre}{\br{\ }{\,}}
\newcommand{\C}{\mathbb{C}}
\newcommand{\F}{\mathbb{F}}
\newcommand{\h}{\lag{h}}
\newcommand{\inp}[2]{\langle #1, #2 \rangle}
\newcommand{\inpe}{\inp{\ }{\,}}
\newcommand{\lag}[1]{\mfrak{#1}}
\newcommand{\mfrak}[1]{\mathfrak{#1}}
\newcommand{\R}{\mathbb{R}}
\renewcommand{\a}{\alpha}
\newcommand{\surj}{\rightarrow\kern-.82em\rightarrow}
\newcommand{\tQ}{\widetilde{Q}}
\renewcommand{\v}{\lal{v}}
\newcommand{\z}{\lal{z}}
\newcommand{\V}{\mathfrak{g}}
\newcommand{\fg}{\mathfrak{g}}
\newcommand{\fz}{\mathfrak{z}}
\newcommand{\fv}{\mathfrak{v}}
\newcommand{\fh}{\mathfrak{h}}
\newcommand{\QQ}{\mathbb{Q}}
\newcommand{\ZZ}{\mathbb{Z}}
\newcommand{\RR}{\mathbb{R}}
\newcommand{\CC}{\mathbb{C}}
\newcommand{\NN}{\mathbb{N}}
\newcommand{\FF}{\mathbb{F}}
\newcommand{\zvec}{\mathbf{0}}


% colored text commands
\newcommand{\red}[1]{{\color{red} #1}}
\newcommand{\grn}[1]{{\color{green} #1}}
\newcommand{\blu}[1]{{\color{blue} #1}}
\newcommand{\ylw}[1]{{\color{yellow} #1}}
\newcommand{\mgn}[1]{{\color{magenta} #1}}
\newcommand{\cyn}[1]{{\color{cyan} #1}}

% tiks packages
\usepackage{tikz}
\usetikzlibrary{calc}
\usepackage{tikzscale}
\usepackage{filecontents}
\usepackage{caption}
\usepackage{subcaption}
\usepackage{wrapfig}

\begin{filecontents*}{heis.tikz}
    \begin{tikzpicture}
        \draw (-4,4) -- (1,2);
        \draw (1, 2) -- (4,4);
        \draw (-1,6) -- (4,4);
        \draw (-1,6) -- (-4,4);
        \draw [-] (0,8) -- (0,4);
        \draw [-] (0,2.4) -- (0,0);
        %\draw [dashed] (4,0) -- (4,3);
        \node [right] at (0, 8) {$\fz$};
        \node [below] at (4,4) {$\fv$};
    \end{tikzpicture}
\end{filecontents*}

\begin{filecontents*}{lStack.tikz}
    \begin{tikzpicture}
%%%%%%%%%%%%%%%%%%%%top%%%%%%%%%%%%%%%%%%%%%%%%%
        \draw (-4,4) -- (1,2);   
        \draw (1, 2) -- (4,4);
        \draw (-1,6) -- (4,4);
        \draw (-1,6) -- (-4,4);
%%%%%%%%%%%%%%%%%%%legs%%%%%%%%%%%%%%%%%%%%%%%%%
        \draw (-4, 4) -- (-4, 1);
        \draw (1, 2) -- (1, -1);
        \draw (-1, 6) -- (-1, 3);
        \draw (4, 4) -- (4, 1); 
%%%%%%%%%%%%%%%%%L_{i, j}%%%%%%%%%%%%%%%%%%%%%%%
        \draw (.58, 4) -- (.58, 1);        %legs
        \draw (1.5, 4) -- (1.5, 1);
        \draw (.95, 4.26) -- (.95, 1.26);
        \draw (1.165, 3.7) -- (1.165, .7);
        \draw (.58, 4) -- (.95, 4.26);      %top
        \draw (1.5, 4) -- (1.165, 3.7);
        \draw (.95, 4.26) -- (1.5, 4);
        \draw (1.165, 3.7) -- (.58, 4);
        \draw (.58, 1) -- (.95, 1.26);   %bottom
        \draw (1.5, 1) -- (1.165, .7);
        \draw (.95, 1.26) -- (1.5, 1);
        \draw (1.165, .7) -- (.58, 1);
%%%%%%%%%%%%%%%%%%%bottom%%%%%%%%%%%%%%%%%%%%%%%
        \draw (-4, 1) -- (1, -1);
        \draw (1, -1) -- (4, 1);
        \draw (-1, 3) -- (4, 1);
        \draw (-1, 3) -- (-4, 1);
    \end{tikzpicture}
\end{filecontents*}

\begin{filecontents*}{lStackNOTOP.tikz}
    \begin{tikzpicture}
%%%%%%%%%%%%%%%%%%%%top%%%%%%%%%%%%%%%%%%%%%%%%%
%        \draw (-4,4) -- (1,2);   
%        \draw (1, 2) -- (4,4);
%        \draw (-1,6) -- (4,4);
%        \draw (-1,6) -- (-4,4);
%%%%%%%%%%%%%%%%%%%legs%%%%%%%%%%%%%%%%%%%%%%%%%
%        \draw (-4, 4) -- (-4, 1);
%        \draw (1, 2) -- (1, -1);
%        \draw (-1, 6) -- (-1, 3);
%        \draw (4, 4) -- (4, 1); 
%%%%%%%%%%%%%%%%%L_{i, j}%%%%%%%%%%%%%%%%%%%%%%%
        \draw (.58, 4) -- (.58, 1);        %legs
        \draw (1.5, 4) -- (1.5, 1);
        \draw (.95, 4.26) -- (.95, 1.26);
        \draw (1.165, 3.7) -- (1.165, .7);
        \draw (.58, 4) -- (.95, 4.26);      %top
        \draw (1.5, 4) -- (1.165, 3.7);
        \draw (.95, 4.26) -- (1.5, 4);
        \draw (1.165, 3.7) -- (.58, 4);
        \draw (.58, 1) -- (.95, 1.26);   %bottom
        \draw (1.5, 1) -- (1.165, .7);
        \draw (.95, 1.26) -- (1.5, 1);
        \draw (1.165, .7) -- (.58, 1);
%%%%%%%%%%%%%%%%%%%bottom%%%%%%%%%%%%%%%%%%%%%%%
        \draw (-4, 1) -- (1, -1);
        \draw (1, -1) -- (4, 1);
        \draw (-1, 3) -- (4, 1);
        \draw (-1, 3) -- (-4, 1);
    \end{tikzpicture}
\end{filecontents*}

\newcommand{\tikzmark}[1]{\tikz[overlay,remember picture] \node (#1) {};}
\newcommand{\DrawBox}[1][]{%
    \tikz[overlay,remember picture]{
    \draw[red,#1]
      ($(left)+(-0.2em,0.9em)$) rectangle
      ($(right)+(0.2em,-0.3em)$);}
}

\NewDocumentCommand{\highlight}{O{blue!40} m m}{%
    \draw[mycolor=#1] (#2.north west)rectangle (#3.south east);
}

\NewDocumentCommand{\fhighlight}{O{blue!40} m m}{%
    \draw[myfillcolor=#1] (#2.north west)rectangle (#3.south east);
}

\begin{document}

\section{Definition and Examples}


\begin{frame}{}
\vspace{1.35 cm}

 \titlepage

\includegraphics[height=1.35cm]{oswego_logo_horiz_grn}

\end{frame}

% \begin{frame}{}
%
% \end{frame}

% \begin{frame}{Outline}
%   \begin{center}
%     \begin{minipage}[]{.85\textwidth}
%       \tableofcontents
%     \end{minipage}
%   \end{center}
% \end{frame}

\subsection{Definition}

\note{So the Lie Algebra is a \emph{vector space}. Does anyone recognize
a similar operator, maybe from calulus or physics, that takes two vectors
and produces a vector?}
\begin{frame}{Definition}
A \emph{\alert{Lie Algebra}} is a (finite-dimensional) vector space $\fg$
together with a bilinear multiplication $\bre : \fg \times \fg \to \fg$
that satisfies three properties: \pause

\smallskip

\begin{enumerate}
\item $\br{x}{x} = 0$ for all $x \in \fg$,
\pause

\

\item $\br{\br{x}{y}}{z} + \br{\br{y}{z}}{x} + \br{\br{z}{x}}{y} = 0$ for
all $x, y, z \in V$.
and
\pause

\

\item  $\br{x+y}{z} = \br{x}{z}+\br{y}{z}\quad \text{and} \quad 
    \br{\a x}{y}=\br{x}{\a    y}=\a\br{x}{y}$ 
\end{enumerate}
\pause

\

Property 1 is known as \emph{\alert{alternating}}, property 2 is known
as the \emph{\alert{Jacobi Identity}} and property 3 is known as 
\emph{\alert{bilinearity}}.
\pause

\

Property 1 and 3 imply another property called \emph{\alert{skew symmetry}}\,:

\smallskip

\begin{enumerate}
\item $\br{x}{y} = -\br{y}{x}$ for all $x, y \in V$. 
\end{enumerate}
\pause 

\

The product $\bre$ is known as a \emph{\alert{Lie bracket}} on $V$.

\end{frame}

\subsection{Examples}

\begin{frame}{skew-symmetry}
By bilinearity, every alternating product is also skew-symmetric,          regardless of the characteristic of the underlying field. Indeed, if [ , ] is alternating then
\pause
        \begin{align}
            0 &= [x + y, x + y] 
            \\&= [x + y, x] + [x + y, y]
            \\&= [x,x] + [y, x] + [x + y, y]
            \\&= [x,x] + [y, x] + [x, y] + [y, y]
            \\0&=         [x, y] + [y, x]
            \\\implies [x,y] &= -[y,x]
        \end{align}
\pause
Conversely, if $\br{}{}$ is skew-symmetric, then
$\br{x}{x}+\br{x}{x}$ = 0 implies that 2[x,x] = 0. Now we see that this implies [x,x] = 0
so long as our field is not of characteristic 2, for in those spaces 2 = 0 and we
can deduce nothing about [x,x].
\end{frame}

%\begin{frame}{Inner Product}
%An \emph{inner product} on the vector $\fg$ is a \emph{symmetric, non-degenerate, bilinear and positive definite}
%function on $\fg$.  That is, an inner product on $\fg$ is a function $\langle\cdot,\cdot\rangle:\fg\times\fg\to\RR$
%that satifies the following properties:
%For $x,y,z \in \fg$
%\pause
%\begin{enumerate}
%    \item $\langle x,y \rangle = \langle y,x \rangle$
%        \pause
%    \item If $\langle x,y \rangle = 0 \quad \forall y \in \fg$
%        \\then $x = \zvec$
%        \pause
%    \item $\langle x+y,z \rangle = \langle x,z \rangle + \langle y,z \rangle$
%        \\and $\a \langle x,y \rangle = \langle \a x,y \rangle = \langle x,\a y \rangle$
%        \pause
%    \item $\langle x,x \rangle \geq 0$
%        \\and $\langle x,x\rangle = 0 \implies x=\zvec$
%        \pause
%\end{enumerate}
%\end{frame}

\begin{frame}{The Center}
    %\begin{definition}
        The \emph{center} of a Lie Algebra, $\fg$ is
        \pause
        \\\[\fz=\{z \in \fg \mid [z,x]
        = \zvec \; \forall x \in \fg\}.\]
        \pause
        \\A vector, $z$, of $\fg$ is said to be in the center of $\fg$ if
        \\\[[ x,z] = \zvec \; \forall x \in \fg.\]
    %\end{definition}
\end{frame}

\begin{frame}{One-Step Nilpotence}
Let $\fg$ be any vector space, and define
$$
\br{x}{y} = 0 \text{\ \ for\ all\ } x,y \in V.
$$
\pause

\

Clearly $\bre$ is alternating:
$$
\br{x}{x} = 0 \text{\ \ for\ all\ } x \in V.
$$
\pause

\

And the Jacobi Identity is trivial:

\pause
\begin{eqnarray*}
\br{\br{x}{y}}{z} + \br{\br{y}{z}}{x} + \br{\br{z}{x}}{y} & = & 
	\br{0}{z} + \br{0}{x} + \br{0}{y} \\
	& = & 0 + 0 + 0 = 0.
\end{eqnarray*}
\pause
\
This Algebra is called \emph{Abelian} or \emph{One-Step Nilpotent}
\pause

\end{frame}

\begin{frame}{$\R^3$ as a non-Abelian Lie Algebra}
Let $x = (x_1, x_2, x_3)$ and $y = (y_1, y_2, y_3)$. The \at{cross product}
of $x$ with $y$ is defined by
$$
x \times y = (x_2y_3 - x_3y_2, x_3y_1 - x_1y_3, x_1y_2 - x_2y_1).
$$
\pause

The cross product is skew symmetric:
$$
x \times y = - y \times x \text{\ \ for\ all\ } x,y \in \R^3.
$$
\pause

Using some identities from Calc III, we can show that the cross product satisfies
the Jacobi Identity:
\pause
\begin{eqnarray*}
& & \hspace{-1.5 cm} (x \times y) \times z + (y \times z) \times x + (z \times x) \times y
\pause
\\ & = & 
\red{(x \cdot z)y} - \blu{(y \cdot z)x} + \grn{(y\cdot x)z} - \red{(z \cdot x)y} +
\blu{(z \cdot y)x} - \grn{(x \cdot y)z}
\pause
\\ & = & 0.
\end{eqnarray*}
\end{frame}

\begin{frame}{$\h_3$ : The Heisenberg Algebra}
Consider a 3-dimensional vector space with basis vectors $x, y, z$.
\pause
Define a Lie bracket on this space by
$$
\br{x}{y} = z,
$$
with all other brackets equal to 0.
\pause
This is called the \at{Heisenberg Algebra}, and is denoted by $\lag{h}_3$.
\pause

\begin{figure}[h]
    \centering
    \includegraphics[width=30mm]{heis.tikz} %input file
    \label{fig:pic} %label name
    \caption{\footnotesize The Heisenberg Algebra, $\fh$} %caption
\end{figure}

\end{frame}

\begin{frame}{$\h_{2k+1}$ : Higher Dimensional Heisenberg Algebras}
Higher odd-dimensional analogues of the Heisenberg algebra can also be defined.
\pause

\

Consider $\R^{2k + 1}$, $k \geq 1$, with basis vectors $x_1, x_2, \hdots, x_k$,
$y_1, y_2, \hdots, y_k$, and $z$.
\pause
Define a Lie bracket on these generators \emph{via}
$$
\br{x_i}{x_j} = 0, \ \ \br{y_i}{y_j} = 0,\ \ \br{x_i}{y_j} = \delta_{ij},
\ \ \text{and}\ \ \br{z}{\ } = 0.
$$
\pause

A general element of this space can be represented by a matrix of the form
$$
X = \begin{pmatrix}
1 & x_1 & x_2 & \cdots & z \\
0 & 1 & 0 & \cdots & y_1 \\
0 & 0 & 1 & \cdots & y_2 \\
\vdots & & & \ddots & \vdots \\
0 & 0 & 0 & \cdots & 1
\end{pmatrix}
$$
\end{frame}

\section{Defining the Lie Algebra}

\subsection{Inner Products}

\begin{frame}{Inner Products}
An \at{inner product} on a vector space $V$ is a non-degenerate, symmetric,
bilinear function $\inpe :V \times V \to \R$.
\pause
\begin{enumerate}
\item If $\inp{x}{y} = 0$ for all $y \in V$, then $x$ must be $0$;
\pause

\item $\inp{x}{y} = \inp{y}{x}$ for all $x,y \in V$;
\pause

\item $\inp{ax + by}{z} = a\inp{x}{z} + b\inp{y}{z}$ for all $x,y,z \in V$, $a, b, \in \R$.
\end{enumerate}
\pause

\

{\bf Example} The \at{dot product} on $\R^3$:
\pause
Let $x, y \in \R^3$; then $x \cdot y = x^Ty$.  The dot product is given by
\pause
\[
    \langle x,y \rangle =  x_1y_1 + x_2y_2 + x_3y_3
\]

\

\end{frame}

\subsection{Encoding Geometry via Algebra}

\begin{frame}{Adjoints of Adjoints}
If you put an inner product on a Lie algebra (rather than just a vector space),
then you will have two different types of products on the space.
\pause

\

The \at{adjoint representation} of a Lie algebra $(\lag{g},\bre)$ on itself is
the function defined by
$$
\ad{x}y = \br{x}{y}.
$$
\pause
Fixing $x \in \lag{g}$, $\ad{x} = \br{x}{\ }$ becomes a function from $\lag{g}$
to itself.
\pause
One can then define a map $j(x) :\lag{g} \to \lag{g}$ by the formula
$$
\inp{\br{x}{y}}{z} = \inp{y}{j(x)z}
$$
for all $y, z \in \lag{g}$.
\pause

\

The map $j(x)$ is called the \at{adjoint} to $\ad{x}$ with respect to the
inner product $\inpe$. (Confusing, I know.)
\end{frame}

\begin{frame}{$j$-maps}
The equation that we used to define $j(x)$,
$$
\inp{\br{x}{y}}{z} = \inp{y}{j(x)z}
$$
could be re-written to look like
\pause
$$
\inp{\ad{x}y}{z} = \inp{y}{\add{x}z}.
$$
\pause
Now we could define $j(x) = \add{x} : \lag{g} \to \lag{g}$.
\pause

\

The $j$-maps are extremely useful because they simultaneously encode information
about the algebraic structure and geometric structure of the Lie algebra $\lag{g}$.
\pause

\

This makes the computation of many complicated geometric objects--such as
curvatures--into ``simple" calculations in linear algebra!
\end{frame}

\begin{frame}{The Lie Bracket as a Matrix with Vector entries}
    In order to do math with Lie Algebras on a computer,
    we would like to represent the Lie Bracket in a matrix form.
    \pause
    Consider the Lie Bracket, with bases $e_1, e_2, \dots, e_n$
    Formally, $L_{ij} = [e_i, e_j]$
    \pause

    $$
     L = \begin{pmatrix}
        L_{11} & L_{12} & \cdots & L_{1n} \\
        L_{21} & L_{22} & \cdots & L_{2n} \\
        \vdots &        & \ddots & \vdots \\
        L_{n1} & L_{n2} & \cdots & L_{nn}
    \end{pmatrix}
    $$
\pause
What is special about $[e_i,e_i]$?
\pause
\\$[e_i,e_i] = \zvec$
\end{frame}

\begin{frame}{The Lie Bracket as a Matrix with Vector entries}
    In order to do math with Lie Algebras on a computer,
    we would like to represent the Lie Bracket in a matrix form.
    Consider the Lie Bracket, with bases $e_1, e_2, \dots, e_n$
    Formally, $L_{ij} = [e_i, e_j]$
    $$
     L = \begin{pmatrix}
        \zvec& L_{12} & \cdots & L_{1n} \\
        L_{21} & \zvec& \cdots & L_{2n} \\
        \vdots &        & \ddots & \vdots \\
        L_{n1} & L_{n2} & \cdots & \zvec
    \end{pmatrix}
    $$
\pause
Given $[e_i, e_j]$, what do we know about $[e_j, e_i]$?
\pause
\\$[e_i, e_j] = -[e_j, e_i]$
\end{frame}
\begin{frame}{The Lie Bracket as a Matrix with Vector entries}
    In order to do math with Lie Algebras on a computer,
    we would like to represent the Lie Bracket in a matrix form.
    Consider the Lie Bracket, with bases $e_1, e_2, \dots, e_n$
    Formally, $L_{ij} = [e_i, e_j]$
    $$
     L = \begin{pmatrix}
        \zvec& L_{12}  & \cdots & L_{1n} \\
        -L_{12}& \zvec & \cdots & L_{2n} \\
        \vdots &       & \ddots & \vdots \\
        -L_{1n}&-L_{2n}& \cdots & \zvec
    \end{pmatrix}
    $$
\end{frame}
\begin{frame}{The Lie Bracket as a Matrix with Vector entries}
    In order to do math with Lie Algebras on a computer,
    we would like to represent the Lie Bracket in a matrix form.
    Consider the Lie Bracket, with bases $e_1, e_2, \dots, e_n$
    Formally, $L_{ij} = [e_i, e_j]$
    $$
     L = \begin{pmatrix}
               \zvec& L_{12}  & \cdots & L_{1n} \\
               -L_{12}& \zvec & \cdots & L_{2n} \\
               \vdots &       & \ddots & \vdots \\
               -L_{1n}&-L_{2n}& \cdots & \zvec
        \end{pmatrix}
    $$
    \\
    Recall from before the adjoint map, $\ad{e_1}{} : \fg \to \fg$ to
    itself.  Can anyone see a matrix representation of $\ad{e1}{}$?
\end{frame}

\begin{frame}{The Lie Bracket as a Matrix with Vector entries}
    In order to do math with Lie Algebras on a computer,
    we would like to represent the Lie Bracket in a matrix form.
    Consider the Lie Bracket, with bases $e_1, e_2, \dots, e_n$
    Formally, $L_{ij} = [e_i, e_j]$
    $$
     L = \begin{pmatrix}
         \tikzmark{left}\zvec& L_{12}  & \cdots & L_{1n} \tikzmark{right}\\
            \DrawBox[thick]
               -L_{12}& \zvec & \cdots & L_{2n} \\
               \vdots &       & \ddots & \vdots \\
               -L_{1n}&-L_{2n}& \cdots & \zvec
        \end{pmatrix}
    $$
    \\
    Recall from before the adjoint map, $\ad{e_1}{} : \fg \to \fg$ to
    itself.  Can anyone see a matrix representation of $\ad{e1}{}$?
\end{frame}
\begin{frame}
    It's nice to see that we can \emph{encode} the bracket into a matrix
    representation, but how can we use that encoding to as the Lie Bracket
    ``map''?
    \\Notice that we can retrieve information about the bases.
    \\$\br{e_1}{e_2} = L_{12}$
    $$
     L = \begin{pmatrix}
         \zvec& \tikzmark{left}L_{12}\tikzmark{right} & \cdots & L_{1n} \\
            \DrawBox[thick]
               -L_{12}& \zvec & \cdots & L_{2n} \\
               \vdots &       & \ddots & \vdots \\
               -L_{1n}&-L_{2n}& \cdots & \zvec
        \end{pmatrix}
    $$
    \\So it will be enough find a way to write all Lie Brackets as a linear
    combination of basis brackets.
\end{frame}

\begin{frame}[Lie Bracket as Linear Combination of Bases]
    We can write $\br{x}{y}$ as a linear combination their component vectors's
    brackets.
    Let $x,y \in \fg$ with
    \\$x=w_1e_1 + w_2e_2$
    \\$y=v_1e_1 + v_2e_2$
    \begin{align}
        \text{So} \br{x}{y} &= \br{w_1e_1 + w_2e_2}{v_1e_1 + v_2e_2}
            \\&= \br{w_1e_1}{v_1e_1 + v_2e_2} + \br{w_2e_2}{v_1e_1 + v_2e_2}
            \\&= \br{w_1e_1}{v_1e_1} + \br{w_1e_1}{v_2e_2}+ \br{w_2e_2}{v_1e_1} + \br{w_2e_2}{v_2e_2}
            \\&= w_1v_1\br{e_1}{e_1} + w_1v_2\br{e_1}{e_2}+ w_2v_1\br{e_2}{e_1} + w_2v_2\br{e_2}{e_2}
            \\&= w_1v_1L_{11} + w_1v_2L_{12} + w_2v_1L_{21}+ w_2v_2L_{22}
    \end{align}
\end{frame}
\note{Now you may be thinking that we have a way of encoding information about
the adjoint maps on the bases, but what about some arbitrary element $x$?}

\section*{}
\subsection*{}

\begin{frame}{TikZ}
\smallskip

\smallskip

\end{frame}

\section*{}
\subsection*{}

\begin{frame}{Acknowledgements}
    \begin{figure}
    \centering
    \includegraphics[width=30mm]{lStack.tikz} %input file
    \label{fig:pic} %label name
    \caption{\footnotesize The L-Stack} %caption
    \end{figure}
    
        \begin{figure}
    \centering
    \includegraphics[width=30mm]{lStackNOTOP.tikz} %input file
    \label{fig:pic} %label name
    \caption{\footnotesize The L-Stack} %caption
    \end{figure}

\end{frame}

\end{document}
