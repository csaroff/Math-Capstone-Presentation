% Matrix Representaion of j-maps slides
% format: latex
% last changed: 1 may 2015

\documentclass{beamer}

\usepackage{amsmath,amsfonts,amssymb,graphicx}

\usetheme{Warsaw}

\newcommand{\br}[2]{\left[#1,#2\right]}
\newcommand{\bre}{\br{\ }{\,}}
\newcommand{\ds}{\oplus}
\newcommand{\inp}[2]{\left\langle #1,#2\right\rangle}
\newcommand{\inpe}{\inp{\ }{\,}}
\newcommand{\lal}[1]{\mathfrak{#1}}
\newcommand{\lan}{\lal{n}}
\newcommand{\lav}{\lal{v}}
\newcommand{\laz}{\lal{z}}
\newcommand{\R}{\mathbb{R}}

\renewcommand{\span}[1]{\text{span}\left\{#1\right\}}

\begin{document}
% You should be able to just copy and paste these frames where you want
% them to go.
\begin{frame}{Matrix representation of the $j$-maps}
Let $\lan = \lav \ds \laz$ be a 2-step nilpotent Lie algebra.
\pause

\

Recall that the Lie bracket on a 2-step nilpotent Lie algebra is a bilinear
map $\bre :\lav \times \lav \to \laz$.
\pause

\


For every $z \in \laz$, one can define a linear transformation 
$j_z :\lav \to \lav$ by the identity
$$
\inp{y}{j_z(x)}_\lav = \inp{z}{\br{x}{y}}_\laz.
$$
\pause

We can use the matrix representations of $\inpe_\lav$ and $\bre$ to find a
matrix representation for $j_z$.
\end{frame}

\begin{frame}{Matrix representation of the $j$-maps}
By linearity, we will know how to construct any $j$-map, if we know the
$j$-maps corresponding to basis vectors of $\laz$. 
\pause

\

Suppose $\laz = \span{z_1,z_2,\hdots,z_m}$.
\pause
Then for any $z_k$, the $j$-map $j_{z_k} :\lav \to \lav$ is given by
\begin{eqnarray*}
\inp{y}{j_{z_k}(x)}_\lav	& = & \inp{z_k}{\br{x}{y}}_\laz \\ \pause
y^T E (J_{z_k}x)		& = & z_k^T (x^TLy) \\ \pause
y^T (E J_{z_k}) x		& = & y^T(L^k)^T x
\end{eqnarray*}
taking advantage of some clever stack-matrix manipulations.
\end{frame}

\begin{frame}{Matrix representation of the $j$-maps}
Since $y^T(EJ_{z_k})x = y^T(L^k)^Tx$ for arbitrary $x$ and $y$ in $\lav$, we
deduce that $EJ_{z_k} = (L^k)^T$.
\pause

\

Since $\det(E) \neq 0$, we may solve for $J_{z_k}$ to obtain
\pause
$$
J_{z_k} = E^{-1}(L^k)^T \in \R^{n \times n}.
$$
\pause

If $z = \zeta_1z_1 + \zeta_2z_2 + \cdots + \zeta_mz_m$, then the map
$j_z$ is represented by the matrix
\pause
$$
J_z = \zeta_1J_{z_1} + \zeta_2J_{z_2} + \cdots + \zeta_mJ_{z_m}.
$$
\end{frame}

\begin{frame}{The matrix $J$ as a stack}
The $j$-maps of a 2-step nilpotent Lie algebra can be described by a stack
of matrices of the same type as $L$.
\pause

\

Indeed, if we let $J^k = J_{z_k}$,
\pause
then
$$
J = \begin{pmatrix}
J^1 \\ J^2 \\ \vdots \\ J^m
\end{pmatrix}
$$
is an $(m \times 1)$ stack of $(n \times n)$ matrices.
\pause
Then for $z = (\zeta_1,\zeta_2,\hdots,\zeta_m)^T \in \laz$, the map $j_z$ is 
represented by
\pause
$$
J_z = z^TJ.
$$
\end{frame}

\begin{frame}{Some Examples}
\begin{center}
{\it to be done...}
\end{center}

\

We wrote some code in Sage to help us compute the $j$-maps for a couple of
interesting Lie algebras.
\end{frame}
\end{document}