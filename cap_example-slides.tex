% Examples of j-maps slides
% format: latex
% last changed: 4 may 2015

\documentclass{beamer}

\usepackage{amsmath,amsfonts,amssymb,graphicx}

\usetheme{Warsaw}

\newcommand{\br}[2]{\left[#1,#2\right]}
\newcommand{\bre}{\br{\ }{\,}}
\newcommand{\ds}{\oplus}
\newcommand{\inp}[2]{\left\langle #1,#2\right\rangle}
\newcommand{\inpe}{\inp{\ }{\,}}
\newcommand{\lal}[1]{\mathfrak{#1}}
\newcommand{\lan}{\lal{n}}
\newcommand{\lav}{\lal{v}}
\newcommand{\laz}{\lal{z}}
\newcommand{\R}{\mathbb{R}}

\renewcommand{\span}[1]{\text{span}\left\{#1\right\}}

\begin{document}
% You should be able to just copy and paste these frames where you want
% them to go.

\begin{frame}{Examples of $j$-maps}
We used Sage to help us compute the $j$-maps for some examples of 2-step nilpotent
Lie algebras.
\pause

\

First, consider the Heisenberg algebra $\lal{h}_3$. Restricting $L$ to $\lav$ and
letting $E$ be the dot product on $\lav$, we obtain
$$
E = \begin{pmatrix}
1 & 0 \\ 0 & 1
\end{pmatrix},\ \ \text{and}\ \ L = \begin{pmatrix}
0 & 1 \\ -1 & 0
\end{pmatrix}.
$$
\pause
The $j$-map $j_{e_3} :\lav \to \lav$ is then given by 
$$
J = E^{-1}L^T = \begin{pmatrix}
0 & -1 \\ 1 & 0
\end{pmatrix}.
$$
This is easy to verify by hand.
\end{frame}

\begin{frame}{Examples of $j$-maps}
Suppose we change the inner product to
$$
E = \begin{pmatrix}
1 & \frac{1}{2} \\[0.5 ex] \frac{1}{2} & 1
\end{pmatrix}.
$$
Then the $j$-map becomes
$$
J = \begin{pmatrix}
-\frac{2}{3} & -\frac{4}{3} \\[0.5 ex] \frac{4}{3} & \frac{2}{3}
\end{pmatrix}.
$$
This is also not too hard to verify by hand.
\end{frame}

\begin{frame}{Examples of $j$-maps}
As a second example, consider the 6-dimensional algebra spanned by the vectors
$\{e_1,e_2,e_3,e_4,z_1,z_2\}$, with non-trivial brackets
\begin{eqnarray*}
\br{e_1}{e_3} & = & z_1 \\
\br{e_2}{e_4} & = & z_1 \\
\br{e_1}{e_4} & = & z_2 \\
\br{e_2}{e_3} & = & z_2
\end{eqnarray*}
The Lie bracket is represented by the matrices
$$
L^1 = \begin{pmatrix}
0 & 0 & 1 & 0 \\
0 & 0 & 0 & 1 \\
-1 & 0 & 0 & 0 \\
0 & -1 & 0 & 0
\end{pmatrix},\ \ \text{and}\ \ L^2 = \begin{pmatrix}
0 & 0 & 0 & 1 \\
0 & 0 & 1 & 0 \\
0 & -1 & 0 & 0 \\
-1 & 0 & 0 & 0 
\end{pmatrix}.
$$
\end{frame}

\begin{frame}{Examples of $j$-maps}
If $E$ is the dot product, then the $j$-maps for this algebra are 
$$
J^1 = \begin{pmatrix}
0 & 0 & -1 & 0 \\
0 & 0 & 0 & -1 \\
1 & 0 & 0 & 0 \\
0 & 1 & 0 & 0
\end{pmatrix},\ \ \text{and}\ \ J^2 = \begin{pmatrix}
0 & 0 & 0 & -1 \\
0 & 0 & -1 & 0 \\
0 & 1 & 0 & 0 \\
1 & 0 & 0 & 0 
\end{pmatrix}.
$$

\

\pause
Again, easy to verify by hand. They're just $(L^1)^T$ and $(L^2)^T$.
\end{frame}

\begin{frame}{Examples of $j$-maps}
However, if we let $E$ be something a little more exotic, like
$$
E = \begin{pmatrix} 
1 & -\frac{1}{4} & \frac{1}{3} & 0 \\[0.5 ex]
-\frac{1}{4} & 1 & \frac{1}{2} & \frac{1}{6} \\[0.5 ex]
\frac{1}{3} & \frac{1}{2} & 1 & 0 \\[0.5 ex]
0 & \frac{1}{6} & 0 & 1
\end{pmatrix},
$$
which has $\det(E) = \frac{607}{1296} > 0$, 
\end{frame}

\begin{frame}{Examples of $j$-maps}
then the $j$-maps become
$$
J^1 = \frac{1}{607}\begin{pmatrix}
-582 & -90 & -936 & -540 \\
-756 & -192 & -540 & -1152 \\
1179 & 126 & 582 & 756 \\
126 & 639 & 90 & 192
\end{pmatrix}, \ \ \text{and}
$$
$$
J^2 = \frac{1}{607} \begin{pmatrix}
-90 & -582 & -540 & -936 \\
-192 & -756 & -1152 & -540 \\
126 & 1179 & 756 & 582 \\
639 & 126 & 192 & 90
\end{pmatrix}.
$$
\pause
You wouldn't want to calculate these by hand!
\end{frame}
\end{document}